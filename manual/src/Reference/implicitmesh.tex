\hypertarget{implicitmesh}{%
\section{ImplicitMesh}\label{implicitmesh}}

The \texttt{implicitmesh} module allows you to build meshes from
implicit functions. For example, the unit sphere could be specified
using the function \texttt{x\^{}2+y\^{}2+z\^{}2-1\ ==\ 0}.

To use the module, first import it:

\begin{lstlisting}
import implicitmesh
\end{lstlisting}

To create a sphere, first create an ImplicitMeshBuilder object with the
implict function you'd like to use:

\begin{lstlisting}
var impl = ImplicitMeshBuilder(fn (x,y,z) x^2+y^2+z^2-1)
\end{lstlisting}

You can use an existing function (or method) as well as an anonymous
function as above.

Then build the mesh,

\begin{lstlisting}
var mesh = impl.build(stepsize=0.25)
\end{lstlisting}

The \texttt{build} method takes a number of optional arguments:

\begin{itemize}

\item
  \texttt{start} - the starting point. If not provided, the value
  Matrix({[}1,1,1{]}) is used.
\item
  \texttt{stepsize} - approximate lengthscale to use.
\item
  \texttt{maxiterations} - maximum number of iterations to use. If this
  limit is exceeded, a partially built mesh will be returned.
\end{itemize}
