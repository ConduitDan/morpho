\hypertarget{meshtools}{%
\section{Meshtools}\label{meshtools}}

The Meshtools package contains a number of functions and classes to
assist with creating and manipulating meshes.

\hypertarget{areamesh}{%
\section{AreaMesh}\label{areamesh}}

This function creates a mesh composed of triangles from a parametric
function. To use it:

\begin{lstlisting}
var m = AreaMesh(function, range1, range2, closed=boolean)
\end{lstlisting}

where

\begin{itemize}

\item
  \texttt{function} is a parametric function that has one parameter. It
  should return a list of coordinates or a column matrix corresponding
  to this parameter.
\item
  \texttt{range1} is the Range to use for the first parameter of the
  parametric function.
\item
  \texttt{range2} is the Range to use for the second parameter of the
  parametric function.
\item
  \texttt{closed} is an optional parameter indicating whether to create
  a closed loop or not. You can supply a list where each element
  indicates whether the relevant parameter is closed or not.
\end{itemize}

To use \texttt{AreaMesh}, import the \texttt{meshtools} module:

\begin{lstlisting}
import meshtools
\end{lstlisting}

Create a square:

\begin{lstlisting}
var m = AreaMesh(fn (u,v) [u, v, 0], 0..1:0.1, 0..1:0.1)
\end{lstlisting}

Create a tube:

\begin{lstlisting}
var m = AreaMesh(fn (u, v) [v, cos(u), sin(u)], -Pi...Pi:Pi/4,
                 -1..1:0.1, closed=[true, false])
\end{lstlisting}

Create a torus:

\begin{lstlisting}
var c=0.5, a=0.2
var m = AreaMesh(fn (u, v) [(c + a*cos(v))*cos(u),
                            (c + a*cos(v))*sin(u),
                            a*sin(v)], 0...2*Pi:Pi/16, 0...2*Pi:Pi/8, closed=true)
\end{lstlisting}

\hypertarget{linemesh}{%
\section{LineMesh}\label{linemesh}}

This function creates a mesh composed of line elements from a parametric
function. To use it:

\begin{lstlisting}
var m = LineMesh(function, range, closed=boolean)
\end{lstlisting}

where

\begin{itemize}

\item
  \texttt{function} is a parametric function that has one parameter. It
  should return a list of coordinates or a column matrix corresponding
  to this parameter.
\item
  \texttt{range} is the Range to use for the parametric function.
\item
  \texttt{closed} is an optional parameter indicating whether to create
  a closed loop or not.
\end{itemize}

To use \texttt{LineMesh}, import the \texttt{meshtools} module:

\begin{lstlisting}
import meshtools
\end{lstlisting}

Create a circle:

\begin{lstlisting}
import constants
var m = LineMesh(fn (t) [sin(t), cos(t), 0], 0...2*Pi:2*Pi/50, closed=true)
\end{lstlisting}

\hypertarget{polyhedronmesh}{%
\section{PolyhedronMesh}\label{polyhedronmesh}}

This function creates a mesh from a polyhedron specification.

\hypertarget{equiangulate}{%
\section{Equiangulate}\label{equiangulate}}

Attempts to equiangulate a mesh.

\hypertarget{meshbuilder}{%
\section{MeshBuilder}\label{meshbuilder}}

The \texttt{MeshBuilder} class simplifies user creation of meshes. To
use this class, begin by creating a \texttt{MeshBuilder} object:

\begin{lstlisting}
var build = MeshBuilder()
\end{lstlisting}

You can then add vertices, edges, etc. one by one using
\texttt{addvertex}, \texttt{addedge} and \texttt{addtriangle}. Each of
these returns an element id:

\begin{lstlisting}
var id1=build.addvertex(Matrix([0,0,0]))
var id2=build.addvertex(Matrix([1,1,1]))
build.addedge([id1, id2])
\end{lstlisting}

Once the mesh is ready, call the \texttt{build} method to construct the
\texttt{Mesh}:

\begin{lstlisting}
var m = build.build()
\end{lstlisting}

\hypertarget{meshrefiner}{%
\section{MeshRefiner}\label{meshrefiner}}

The \texttt{MeshRefiner} class is used to refine meshes, and to correct
associated data structures that depend on the mesh.
