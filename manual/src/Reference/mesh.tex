\hypertarget{mesh}{%
\section{Mesh}\label{mesh}}

The \texttt{Mesh} class provides support for meshes. Meshes may consist
of different kinds of element, including vertices, line elements, facets
or area elements, tetrahedra or volume elements.

To create a mesh, you can import it from a file:

\begin{lstlisting}
var m = Mesh("sphere.mesh")
\end{lstlisting}

or use one of the functions available in \texttt{meshtools} or
\texttt{implicitmesh} packages.

Each type of element is referred to as belonging to a different
\texttt{grade}. Point-like elements (vertices) are \emph{grade 0};
line-like elements (edges) are \emph{grade 1}; area-like elements
(facets; triangles) are \emph{grade 2} etc.

The \texttt{plot} package includes functions to visualize meshes.

\hypertarget{save}{%
\section{Save}\label{save}}

Saves a mesh as a .mesh file.

\begin{lstlisting}
m.save("new.mesh")
\end{lstlisting}

\hypertarget{vertexposition}{%
\section{Vertexposition}\label{vertexposition}}

Retrieves the position of a vertex given an id:

\begin{lstlisting}
print m.vertexposition(id)
\end{lstlisting}

\hypertarget{setvertexposition}{%
\section{Setvertexposition}\label{setvertexposition}}

Sets the position of a vertex given an id and a position vector:

\begin{lstlisting}
print m.setvertexposition(1, Matrix([0,0,0]))
\end{lstlisting}

\hypertarget{addgrade}{%
\section{Addgrade}\label{addgrade}}

Adds a new grade to a mesh. This is commonly used when, for example, a
mesh file includes facets but not edges. To add the missing edges:

\begin{lstlisting}
m.addgrade(1)
\end{lstlisting}

\hypertarget{addsymmetry}{%
\section{Addsymmetry}\label{addsymmetry}}

Adds a symmetry to a mesh. Experimental in version 0.5.

\hypertarget{maxgrade}{%
\section{Maxgrade}\label{maxgrade}}

Returns the highest grade element present:

\begin{lstlisting}
print m.maxgrade()
\end{lstlisting}

\hypertarget{count}{%
\section{Count}\label{count}}

Counts the number of elements. If no argument is provided, returns the
number of vertices. Otherwise, returns the number of elements present of
a given grade:

\begin{lstlisting}
print m.count(2) // Returns the number of area-like elements. 
\end{lstlisting}
