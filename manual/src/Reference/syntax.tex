\hypertarget{syntax}{%
\section{Syntax}\label{syntax}}

Morpho provides a flexible object oriented language similar to other
languages in the C family (like C++, Java and Javascript) with a
simplified syntax.

Morpho programs are stored as plain text with the .morpho file
extension. A program can be run from the command line by typing

\begin{lstlisting}
morpho program.morpho
\end{lstlisting}

\hypertarget{comments}{%
\section{Comments}\label{comments}}

Two types of comment are available. The first type is called a `line
comment' whereby text after \texttt{//} on the same line is ignored by
the interpreter.

\begin{lstlisting}
a.dosomething() // A comment
\end{lstlisting}

Longer `block' comments can be created by placing text between
\texttt{/*} and \texttt{*/}. Newlines are ignored

\begin{lstlisting}
/* This
   is
   a longer comment */
\end{lstlisting}

In contrast to C, these comments can be nested

\begin{lstlisting}
/* A nested /* comment */ */
\end{lstlisting}

enabling the programmer to quickly comment out a section of code.

\hypertarget{symbols}{%
\section{Symbols}\label{symbols}}

Symbols are used to refer to named entities, including variables,
classes, functions etc. Symbols must begin with a letter or underscore
\_ as the first character and may include letters or numbers as the
remainder. Symbols are case sensitive.

\begin{lstlisting}
asymbol
_alsoasymbol
another_symbol
EvenThis123
YET_ANOTHER_SYMBOL
\end{lstlisting}

Classes are typically given names with an initial capital letter.
Variable names are usually all lower case.

\hypertarget{newlines}{%
\section{Newlines}\label{newlines}}

Strictly, morpho ends statements with semicolons like C, but in practice
these are usually optional and you can just start a new line instead.
For example, instead of

\begin{lstlisting}
var a = 1; // The ; is optional
\end{lstlisting}

you can simply use

\begin{lstlisting}
var a = 1
\end{lstlisting}

If you want to put several statements on the same line, you can separate
them with semicolons:

\begin{lstlisting}
var a = 1; print a
\end{lstlisting}

There are a few edge cases to be aware of: The morpho parser works by
accepting a newline anywhere it expects to find a semicolon. To split a
statement over multiple lines, signal to morpho that you plan to
continue by leaving the statement unfinished. Hence, do this:

\begin{lstlisting}
print a +
      1
\end{lstlisting}

rather than this:

\begin{lstlisting}
print a   // < Morpho thinks this is a complete statement
      + 1 // < and so this line will cause a syntax error
\end{lstlisting}

\hypertarget{booleans}{%
\section{Booleans}\label{booleans}}

Comparison operations like \texttt{==}, \texttt{\textless{}} and
\texttt{\textgreater{}=} return \texttt{true} or \texttt{false}
depending on the result of the comparison. For example,

\begin{lstlisting}
print 1==2
\end{lstlisting}

prints \texttt{false}. The constants \texttt{true} or \texttt{false} are
provided for you to use in your own code:

\begin{lstlisting}
return true
\end{lstlisting}

\hypertarget{nil}{%
\section{Nil}\label{nil}}

The keyword \texttt{nil} is used to represent the absence of an object
or value.

Note that in \texttt{if} statements, a value of \texttt{nil} is treated
like \texttt{false}.

\begin{lstlisting}
if (nil) {
    // Never executed.
}
\end{lstlisting}

\hypertarget{blocks}{%
\section{Blocks}\label{blocks}}

Code is divided into \emph{blocks}, which are delimited by curly
brackets like this:

\begin{lstlisting}
{
  var a = "Hello"
  print a
}
\end{lstlisting}

This syntax is used in function declarations, loops and conditional
statements.

Any variables declared within a block become \emph{local} to that block,
and cannot be seen outside of it. For example,

\begin{lstlisting}
var a = "Foo"
{
  var a = "Bar"
  print a
}
print a
\end{lstlisting}

would print ``Bar'' then ``Foo''; the version of \texttt{a} inside the
code block is said to \emph{shadow} the outer version.

\hypertarget{precedence}{%
\section{Precedence}\label{precedence}}

Precedence refers to the order in which morpho evaluates operations. For
example,

\begin{lstlisting}
print 1+2*3
\end{lstlisting}

prints \texttt{7} because \texttt{2*3} is evaluated before the addition;
the operator \texttt{*} is said to have higher precedence than
\texttt{+}.

You can always modify the order of evaluation by using brackets:

\begin{lstlisting}
print (1+2)*3 // prints 9
\end{lstlisting}

\hypertarget{print}{%
\section{Print}\label{print}}

The \texttt{print} keyword is used to print information to the console.
It can be followed by any value, e.g.

\begin{lstlisting}
print 1
print true
print a
print "Hello"
\end{lstlisting}
