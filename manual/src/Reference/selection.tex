\hypertarget{selection}{%
\section{Selection}\label{selection}}

The Selection class enables you to select components of a mesh for later
use. You can supply a function that is applied to the coordinates of
every vertex in the mesh, or select components like boundaries.

Create an empty selection: var s = Selection(mesh)

Select vertices above the z=0 plane using an anonymous function: var s =
Selection(mesh, fn (x,y,z) z\textgreater0)

Select the boundary of a mesh: var s = Selection(mesh, boundary=true)

Selection objects can be composed using set operations: var s =
s1.union(s2)

or var s = s1.intersection(s2)

To add additional grades, use the addgrade method. For example, to add
areas: s.addgrade(2)

\hypertarget{addgrade}{%
\section{addgrade}\label{addgrade}}

Adds elements of the specified grade to a Selection. For example, to add
edges to an existing selection, use

\begin{lstlisting}
s.addgrade(1)
\end{lstlisting}

By default, this only adds an element if \emph{all} vertices in the
element are currently selected. Sometimes, it's useful to be able to add
elements for which only some vertices are selected. The optional
argument \texttt{partials} allows you to do this:

\begin{lstlisting}
s.addgrade(1, partials=true)
\end{lstlisting}

Note that this method modifies the existing selection, and does not
generate a new Selection object.

\hypertarget{removegrade}{%
\section{removegrade}\label{removegrade}}

Removes elements of the specified grade from a Selection. For example,
to remove edges from an existing selection, use

\begin{lstlisting}
s.removegrade(1)
\end{lstlisting}

Note that this method modifies the existing selection, and does not
generate a new Selection object.

\hypertarget{idlistforgrade}{%
\section{idlistforgrade}\label{idlistforgrade}}

Returns a list of element ids included in the selection.

To find out which edges are selected: var edges = s.idlistforgrade(1)

\hypertarget{isselected}{%
\section{isselected}\label{isselected}}

Checks if an element id is selected, returning \texttt{true} or
\texttt{false} accordingly.

To check if edge number 5 is selected: var f = s.isselected(1, 5))
