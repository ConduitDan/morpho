\hypertarget{field}{%
\section{Field}\label{field}}

Fields are used to store information, including numbers or matrices,
associated with the elements of a \texttt{Mesh} object.

You can create a \texttt{Field} by applying a function to each of the
vertices,

\begin{lstlisting}
var f = Field(mesh, fn (x, y, z) x+y+z)
\end{lstlisting}

or by supplying a single constant value,

\begin{lstlisting}
var f = Field(mesh, Matrix([1,0,0]))
\end{lstlisting}

Fields can then be added and subtracted using the \texttt{+} and
\texttt{-} operators.

To access elements of a \texttt{Field}, use index notation:

\begin{lstlisting}
print f[grade, element, index]
\end{lstlisting}

where * \texttt{grade} is the grade to select * \texttt{element} is the
element id * \texttt{index} is the element index

As a shorthand, it's possible to omit the grade and index; these are
then both assumed to be \texttt{0}:

\begin{lstlisting}
print f[2]
\end{lstlisting}

\hypertarget{grade}{%
\section{Grade}\label{grade}}

To create fields that include grades other than just vertices, use the
\texttt{grade} option to \texttt{Field}. This can be just a grade index,

\begin{lstlisting}
var f = Field(mesh, 0, grade=2)
\end{lstlisting}

which creates an empty field with \texttt{0} for each of the facets of
the mesh \texttt{mesh}.

You can store more than one item per element by supplying a list to the
\texttt{grade} option indicating how many items you want to store on
each grade. For example,

\begin{lstlisting}
var f = Field(mesh, 1.0, grade=[0,2,1])
\end{lstlisting}

stores two numbers on the line (grade 1) elements and one number on the
facets (grade 2 elements). Each number in the field is initialized to
the value \texttt{1.0}.

\hypertarget{shape}{%
\section{Shape}\label{shape}}

The \texttt{shape} method returns a list indicating the number of items
stored on each element of a particular grade. This has the same format
as the list you supply to the \texttt{grade} option of the
\texttt{Field} constructor. For example,

\begin{lstlisting}
[1,0,2]
\end{lstlisting}

would indicate one item stored on each vertex and two items stored on
each facet.

\hypertarget{op}{%
\section{Op}\label{op}}

The \texttt{op} method applies a function to every item stored in a
\texttt{Field}. For example,

\begin{lstlisting}
f.op(fn (x) x.norm())
\end{lstlisting}

calls the \texttt{norm} method on each element stored in \texttt{f}.

Additional \texttt{Field} objects may supplied as extra arguments to
\texttt{op}. These must have the same shape (the same number of items
stored on each grade). The function supplied to \texttt{op} will now be
called with the corresponding element from each field as arguments. For
example,

\begin{lstlisting}
f.op(fn (x,y) x.inner(y), g)
\end{lstlisting}

calculates an elementwise inner product between the elements of Fields
\texttt{f} and \texttt{g}, returning the result as elements of a new
\texttt{Field} object.
