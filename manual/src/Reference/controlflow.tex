\hypertarget{control-flow}{%
\section{Control Flow}\label{control-flow}}

Control flow statements are used to determine whether and how many times
a selected piece of code is executed. These include:

\begin{itemize}

\item
  \texttt{if} - Selectively execute a piece of code if a condition is
  met.
\item
  \texttt{else} - Execute a different block of code if the test in an
  \texttt{if} statement fails.
\item
  \texttt{for} - Repeatedly execute a section of code with a counter
\item
  \texttt{while} - Repeatedly execute a section of code while a
  condition is true.
\end{itemize}

\hypertarget{if}{%
\section{If}\label{if}}

If allows you to selectively execute a section of code depending on
whether a condition is met. The simplest version looks like this:

\begin{lstlisting}
if (x<1) print x
\end{lstlisting}

where the body of the loop, \texttt{print\ x}, is only executed if x is
less than 1. The body can be a code block to accommodate longer sections
of code:

\begin{lstlisting}
if (x<1) {
    ... // do something
}
\end{lstlisting}

If you want to choose between two alternatives, use \texttt{else}:

\begin{lstlisting}
if (a==b) {
    // do something
} else {
    // this code is executed only if the condition is false
}
\end{lstlisting}

You can even chain multiple tests together like this:

\begin{lstlisting}
if (a==b) {
    // option 1
} else if (a==c) {
    // option 2
} else {
    // something else
}
\end{lstlisting}

\hypertarget{while}{%
\section{While}\label{while}}

While loops repeat a section of code while a condition is true. For
example,

\begin{lstlisting}
var k=1
while (k <= 4) { print k; k+=1 }
       ^cond   ^body
\end{lstlisting}

prints the numbers 1 to 4. The loop has two sections: \texttt{cond} is
the condition to be executed and \texttt{body} is the section of code to
be repeated.

Simple loops like the above example, especially those that involve
counting out a sequence of numbers, are more conveniently written using
a \texttt{for} loop,

\begin{lstlisting}
for (k in 1..4) print k
\end{lstlisting}

Where \texttt{while} loops can be very useful is where the state of an
object is being changed in the loop, e.g.

\begin{lstlisting}
var a = List(1,2,3,4)
while (a.count()>0) print a.pop()
\end{lstlisting}

which prints 4,3,2,1.

\hypertarget{do}{%
\section{Do}\label{do}}

A \texttt{do}\ldots{}\texttt{while} loop repeats code while a condition
is true---similar to a \texttt{while} loop---but the test happens at the
end:

\begin{lstlisting}
var k=1
do {
  print k;
  k+=1
} while (k<5)
\end{lstlisting}

which prints 1,2,3,4

Hence this type of loop executes at least one interation

\hypertarget{for}{%
\section{For}\label{for}}

For loops allow you to repeatedly execute a section of code. They come
in two versions: the simpler version looks like this,

\begin{lstlisting}
for (var i in 1..5) print i
\end{lstlisting}

which prints the numbers 1 to 5 in turn. The variable \texttt{i} is the
\emph{loop variable}, which takes on a different value each iteration.
\texttt{1..5} is a range, which denotes a sequence of numbers. The
\emph{body} of the loop, \texttt{print\ i}, is the code to be repeatedly
executed.

Morpho will implicitly insert a \texttt{var} before the loop variable if
it's missing, so this works too:

\begin{lstlisting}
for (i in 1..5) print i
\end{lstlisting}

If you want your loop variable to count in increments other than 1, you
can specify a stepsize in the range:

\begin{lstlisting}
for (i in 1..5:2) print i
               ^step
\end{lstlisting}

Ranges need not be integer:

\begin{lstlisting}
for (i in 0.1..0.5:0.1) print i
\end{lstlisting}

You can also replace the range with other kinds of collection object to
loop over their contents:

\begin{lstlisting}
var a = Matrix([1,2,3,4])
for (x in a) print x
\end{lstlisting}

Morpho iterates over the collection object using an integer
\emph{counter variable} that's normally hidden. If you want to know the
current value of the counter (e.g.~to get the index of an element as
well as its value), you can use the following:

\begin{lstlisting}
var a = [1, 2, 3]
for (x, i in a) print "${i}: ${x}"
\end{lstlisting}

Morpho also provides a second form of \texttt{for} loop similar to that
in C:

\begin{lstlisting}
for (var i=0; i<5; i+=1) { print i }
     ^start   ^test ^inc.  ^body
\end{lstlisting}

which is executed as follows: start: the variable \texttt{i} is declared
and initially set to zero. test: before each iteration, the test is
evaluated. If the test is \texttt{false}, the loop terminates. body: the
body of the loop is executed. inc: the variable \texttt{i} is increased
by 1.

You can include any code that you like in each of the sections.

\hypertarget{break}{%
\section{Break}\label{break}}

Break is used inside loops to finish the loop early. For example

\begin{lstlisting}
for (i in 1..5) {
    if (i>3) break // --.
    print i        //   | (Once i>3)
}                  //   |
...                // <-'
\end{lstlisting}

would only print 1,2 and 3. Once the condition \texttt{i\textgreater{}3}
is true, the \texttt{break} statement causes execution to continue after
the loop body.

Both \texttt{for} and \texttt{while} loops support break.

\hypertarget{continue}{%
\section{Continue}\label{continue}}

Continue is used inside loops to skip over the rest of an iteration. For
example

\begin{lstlisting}
for (i in 1..5) {     // <-.
    print "Hello"          |
    if (i>3) continue // --'
    print i
}                     
\end{lstlisting}

prints ``Hello'' five times but only prints 1,2 and 3. Once the
condition \texttt{i\textgreater{}3} is true, the \texttt{continue}
statement causes execution to transfer to the start of the loop body.

Traditional \texttt{for} loops also support \texttt{continue}:

\begin{lstlisting}
                // v increment
for (var i=0; i<5; i+=1) {
    if (i==2) continue
    print i
}
\end{lstlisting}

Since \texttt{continue} causes control to be transferred \emph{to the
increment section} in this kind of loop, here the program prints 0..4
but the number 2 is skipped.

Use of \texttt{continue} with \texttt{while} loops is possible but isn't
recommended as it can easily produce an infinite loop!

\begin{lstlisting}
var i=0
while (i<5) {
    if (i==2) continue
    print i
    i+=1
}
\end{lstlisting}

In this example, when the condition \texttt{i==2} is \texttt{true},
execution skips back to the start, but \texttt{i} \emph{isn't}
incremented. The loop gets stuck in the iteration \texttt{i==2}.
