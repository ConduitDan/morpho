\hypertarget{matrix}{%
\section{Matrix}\label{matrix}}

The Matrix class provides support for matrices. A matrix can be
initialized with a given size,

\begin{lstlisting}
var a = Matrix(nrows,ncols)
\end{lstlisting}

where all elements are initially set to zero. Alternatively, a matrix
can be created from an array,

\begin{lstlisting}
var a = Matrix([[1,2], [3,4]])
\end{lstlisting}

You can create a column vector like this,

\begin{lstlisting}
var v = Matrix([1,2])
\end{lstlisting}

Once a matrix is created, you can use all the regular arithmetic
operators with matrix operands, e.g.

\begin{lstlisting}
a+b
a*b
\end{lstlisting}

The division operator is used to solve a linear system, e.g.

\begin{lstlisting}
var a = Matrix([[1,2],[3,4]])
var b = Matrix([1,2])

print b/a
\end{lstlisting}

yields the solution to the system a*x = b.
